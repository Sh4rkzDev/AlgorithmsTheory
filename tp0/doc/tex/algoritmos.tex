\section{Algoritmo para encontrar el tiempo optimo}

A continuacion se detallan el codigo y los pasos que se siguieron para llevar a cabo el algoritmo planteado.

\subsection{Obtener los datos}

Al inciar el programa, se proporciona la ruta de un archivo con los datos a utilizar. Se implemento el siguiente codigo:

\lstinputlisting[language=Python]{code/input.py}

La complejidad del algoritmo propuesto para procesar toda la data es $\mathcal{O}(n)$, ya que solo tiene que procesar todas las lineas del archivo.

\subsection{Obtener el tiempo}

Una vez que se obtienen los datos, se los ordena segun el tiempo de los ayudantes de mayor a menor, y luego podemos conseguir el tiempo total que se tardara en analizar todos los rivales:

\lstinputlisting[language=Python]{code/scaloneta.py}

La complejidad del algoritmo para ordenar todos los datos es $\mathcal{O}\left(n \log n\right)$, ya que el metodo \textit{sort} de Python tiene esa complejidad (\href{https://www.geeksforgeeks.org/sort-in-python/}{complejidad del metodo \textit{sort}}) y la funcion auxiliar que usa como \textit{key} para ordenar es $\mathcal{O}(1)$. Luego, recorrer todo los datos ya ordenados e ir procesando la informacion es $\mathcal{O}(n)$, ya que solo recorre el arreglo y va sumando los valores correspondientes a las variables declaradas, y eso es $\mathcal{O}(1)$. La complejidad final es $\mathcal{O}\left(n \log n + n\right)$

La variable \textit{total} se utiliza para ir contabilizando el tiempo que tarda Scaloni, sumando el tiempo del video que este analizando en el momento mas todos los ya analizados. A su vez, tenemos la variable \textit{actual}, que se encarga de guardar el tiempo que va a requerir el video que se esta analizando actualmente, que es la suma del tiempo que se lleva en total mas la que vaya a tardar el ayudante. Y por ultimo tenemos la variable \textit{longest}, que se encarga de almacenar el video que vaya a tardar mas en analizarse. Es necesaria ya que nada nos garantiza que el ultimo ayudante vaya a ser el que mas influencie en el tiempo total de los analisis de los videos. Un ejemplo para mostrar esto es ir a la exageracion: Que el primer video de todos, el ayudante tarde 5 meses. Por mas que el ultimo tarde 2hs, la duracion total va a ser lo que tarde el analisis que mas se aleje de los ya terminados, tanto por Scaloni y los demas ayudantes.