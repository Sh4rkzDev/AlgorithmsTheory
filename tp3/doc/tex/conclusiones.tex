\section{Conclusiones}

Como se pudo observar, los algoritmos tuvieron sus pros y sus contras. Por un lado, el algoritmo \textit{Greedy} fue de los que más rápido corrió. Esto hace honor también a la característica de los algoritmos Greedy y sus reglas sencillas que usan para conseguir los óptimos locales. También puede ser más personalizado, y acercarlo o alejarlo de la solución óptima, dependiendo de la regla que se siga.

Por otro lado tenemos el algoritmo planteado por \textit{Programación Lineal}. Su velocidad no fue de esperar, y fue sorpresiva para nosotros, a la vez que entregaba la solución óptima en tan corto tiempo. Esto puede decir que usar este tipo de metodología tiene sus ventajas aunque su uso no sea uno de los más cotidianos. Aunque una pequeña observación a tener en cuenta es que a pesar de que su complejidad sea exponencial, los resultados de los tiempos parecen demostrar que no es así. Esto pudo ser debido a varios factores, como por ejemplo que el orden en el que se analizaron las variables internamente haya sido de los mejores casos, o que aunque sea una complejidad exponencial, esta metodología optimice bastante los pasos. Así como el backtracking optimiza a niveles bestiales a comparación con Fuerza Bruta, la Programación Lineal habría que ver que hace por dentro, ya que en realidad se utiliza el paquete de \textit{PulP} para su funcionamiento. A su vez, podemos observar como el algoritmo aproximado cuanto más aumentaban los $n$, es decir, los subconjuntos, mas aumentaba el \textit{ratio de aproximación}, por lo que a mayores valores, mayor será la diferencia entre la solución dada y la óptima.

Y finalmente tenemos el algoritmo por \textit{Backtracking}. Por lejos fue el que más tardó, y demuestra que se cumple la complejidad que caracteriza a dichos algoritmos. Pero también puede tener sus ventajas: si en algún momento se solicita devolver todos las soluciones óptimas de manera que tengas distintas posibilidades de planteles y seguir alegrando a toda la prensa, bastará con modificar algunas líneas de código y nada más. A su vez, al utilizar condiciones de poda, se evita que se vuelva un algoritmo de Fuerza Bruta y aunque la complejidad computacional no cambia, a nivel práctico puede cambiar drásticamente, haciendo que no termine por un largo rato.

Finalmente, por una última vez, valoramos la última oportunidad que se nos brindó de colaborar en esta tarea crucial para que Scaloni pueda alegrar a la prensa y darle oportunidad a posibles nuevos jugadores. Estamos satisfechos por haber contribuido al éxito continuo de la selección \textbf{CAMPEONA DEL MUNDO}, y que esto nos lleve a ganar la \textbf{4ta} en 2026. Se anula todo tipo de mufa.