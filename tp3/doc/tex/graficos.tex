\section{Mediciones}

Se realizaron mediciones en base a distintos sets de pruebas que fueron desde 5 elementos hasta 200 elementos. Para cada set, se ejecutó el algoritmo por \textbf{Backtracking}, el algoritmo utilizando \textbf{Programación Lineal}, y por último el algoritmo por \textbf{Greedy}.

\begin{figure}[H]
	\centering
	\includegraphics[width=0.9\textwidth]{img/graphic.png}
\end{figure}

Como se puede apreciar, el algoritmo planteado por \textit{Backtracking} es el algoritmo que más tarda con diferencia, haciendo honor a su complejidad respecto a los otros dos. Por detrás de él, está el algoritmo usando la técnica \textit{Greedy}, la cual puede variar su complejidad dependiendo que tan enfocado esté en conseguir el óptimo y en que se base su regla para obtener los óptimos. Esto puede variar tanto su complejidad como qué tan lejos del resultado óptimo se encuentre. Y por último, el algoritmo que más rápido parece haber logrado con diferencia fue el usado con \textit{Programación Lineal}. No solo ha conseguido ser el más rápido, sino que a su vez logró dar la respuesta óptima al problema. Observar que el algoritmo aproximado por \textit{Programacion Lineal} es despreciable en cuanto a lo que tardo, al menos lo que se puede observar en el grafico. Pero los resultados en cuanto a la optimalidad de la solucion difieren cada vez mas en cuanto los subconjuntos son mayores.
