\section{Introduccion}

El problema que se nos plantea es que el tiempo total para poder analizar todos los siguientes \textbf{\textit {n}} rivales del \textbf{CAMPEÓN DEL MUNDO} sea el más óptimo. A su vez, se solicita que sea hallado con un algoritmo Greedy. Para tener en mente, los algoritmos Greedy siguen una regla sencilla que les permiten obtener un \textit{óptimo local} según el estado actual del programa, y poder llegar a un \textit{óptimo general} juntando los locales. Pero como todo, pueden tener sus ventajas y desventajas, como por ejemplo que no siempre dan el resultado óptimo, o que demostrar que el resultado es óptimo es difícil. Por otro lado, son intuitivos de pensar y fácil de entender, y suelen ser rápidos. Lo que en reglas generales suele indicar que el algoritmo es Greedy es el uso de colas de prioridad como el \textit{heap} u ordenamientos (recalcar que un algoritmo puede hacer uso de heaps u ordenamientos y no ser Greedy).

Dicho esto, empezamos a plantear posibles soluciones para el problema. Lo primero que se nos vino a la idea es hacer uso de algún tipo de ordenamiento, ya sea ordenando por los tiempos de Scaloni o los ayudantes, en relación a cuánto tardaría cada uno. Lo que nos ayudó a volcarnos por el lado de ordenar por los tiempos de los ayudantes fue el siguiente:
\begin{itemize}
	\item El tiempo total que tarda Scaloni siempre va a ser el mismo, no importa cómo ordenemos los videos a ver. Bien como se dice, \textit{"El orden de los factores no altera el producto"}.
	\item Por más que Scaloni termine de ver el último video del último rival, va a quedar que después un ayudante analice el video, por lo que es más importante el tiempo que va a tardar este último ayudante que el que va a tardar Scaloni.
\end{itemize}

Ahora bien, ya tenemos definido por donde queremos encarar el problema, pero todavía falta definir en qué orden queremos que los videos se visualicen dependiendo de los ayudantes, si los más rápidos primero o viceversa. Acá entra en juego un factor muy importante a tener en cuenta: \textbf{los ayudantes analizan los videos inmediatamente termina Scaloni de ver el video, y el análisis de cada ayudante es independiente a los otros}. Esto quiere decir que si Scaloni terminó un video, inmediatamente uno de los ayudantes se pondrá a analizarlo. Y si Scaloni termina otro video, otro ayudante podrá empezar a analizar ese video, no importa si el anterior terminó de realizar su análisis o no. Esto nos ha llevado a tomar la decisión de ordenar por el tiempo de los ayudantes del que más tarde al que menos, por los siguientes puntos:
\begin{itemize}
	\item Los ayudantes pueden analizar un video independientemente de si el anterior haya terminado o no su análisis.
	\item Si para el último video queda el ayudante que mas tarda, el tiempo total no seria el optimo sino todo lo contrario, ya que se tardaría el tiempo total de Scaloni más lo que tarde este último.
	\item Por ese motivo, conviene que los ayudantes que más tarden estén al principio, ya que tienen tiempo hasta que Scaloni termine todos los videos para terminar. Y los ayudantes que menos tardan, estarán al final, de modo que si Scaloni termina, los que les falten sabemos que son los más rápidos y terminan lo antes posible.
\end{itemize}
