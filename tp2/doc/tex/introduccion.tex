\section{Introduccion}

Luego de haber ayudado a Scaloni y el equipo tecnico a ordenar los analisis de los siguientes rivales de la seleccion \textbf{CAMPEONA DEL MUNDO}, Scaloni planteo un cronograma de entrenamiento para los jugadores, pero se cruzo con un problema: no sabe que dias conviene entrenar y cuales descansar para que los jugadores tengan la mayor ganancia posible y puedan dar el mejor desempenio en el proximo mundial. Pero que no cunda el panico, Menotti le recomendo usar una tecnica que sera la que nos ayudara a resolver este problema: \textit{"Programacion Dinamica"}.

Ahora bien, ¿En qué consiste la \textit{Programacion Dinamica}? Para entender esto, es necesario entender ciertos puntos:
\begin{itemize}
    \item Usa una tecnica de optimizacion llamada \textit{Memoization} (memorizacion en ingles). Esta tecnica se basa en guardar los resultados ya calculados de un problema para poder ser reutilizados mas adelante sin ser necesario que se calculen nuevamente.
    \item El problema que se desea solucioinar debe poder descomponerse en subproblemas que a su vez estos permitan construir las soluciones a problemas mas grandes.
    \item La cantidad de subuproblemas debe ser polinomial.
    \item Esta tecnica nos permite reducir la complejidad temporal de un algoritmo, evitando explorar un espacio exponencial de soluciones.
\end{itemize}
Una vez planteado esto, lo que se refiere a la \textit{Programacion Dinamica} es hacer uso de la \textit{Memoization} y guardar las soluciones de los subproblemas mas pequenios para poder ir construyendo las soluciones cada vez mas grandes hasta llegar a dar con la solucion deseada. Pero para poder aplicar esta tecnica, es necesario antes saber la forma que tienen los subproblemas y como estos se combinan para dar solucion a un problema mas grande. 
Viendo por donde viene el asunto, se nos puede venir a la cabeza la \textit{recurrencia}, y esto es correcto ya que esta tecnica se sustenta gracias al uso de este metodo, y la solucion se hallara gracias a haber hallado la \textbf{\textit{Ecuacion de Recurrencia}} del problema. Una vez obtenida, se puede plantear el problema iterativamente (en vez de hacerlo recursivo) pero siguiendo los principios de la tecnica y hacer uso de la memorizacion. Hacerla de forma iterativa nos sera de gran ayuda ya que es mas facil de entender que es lo que esta pasando y cuando, y porque es mucho mas facil de calcular la complejidad.

En nuestro problema que se nos planteo, necesitamos saber que dias conviene entrenar y cuales no dependiendo de la ganancia obtenida del entrenamiento y la energia de los jugadores. Tener en cuenta lo siguiente:
\begin{itemize}
    \item Para el dia del entrenamiento $e_i$, la energia disponible va a ser menor o igual a la del dia $e_{i-1}$, por lo tanto se cumple
    \[ s_1 >= s_2 >= ... >= s_n \]
    siendo estas las energias disponibles para cada $e_i$ despues de haber entrenado consecutivamente. La energia $s_1$ corresponde al dia $e_1$.
    \item Si se descansa un dia, el primer dia de volver a entrenar vuelve a tener la energia del primer dia.
    \item Si el valor del entrenamiento del dia $e_i$ tiene valor $j$, y la energia disponible para ese dia es $s_i$, la ganancia sera del minimo de estos dos. Es decir,
    \[ Ganancia(i) = min(j, s_i) \]
\end{itemize}
Entonces teniendo en mente lo anterior mencionado, nuestros subproblemas serian proporcionales a la cantidad de dias que hayan programados. A su vez, la forma de estos subproblemas es calcular el optimo del dia \texttt{i} dependiendo de si conviene haber entrenado o no el dia anterior y tener en cuenta la ganancia de este dia. La forma en la que los subproblemas se combinan para dar lugar a la solucion de los problemas mas grandes es a traves de, una vez calculado los \textit{optimos} de los dias hasta \textit{i-1} y teniendo en cuenta descansos y sus ganancias, para el dia \textit{i} la solucion sera hacer uso de la memorizacion de lo ya calculado y elegir la mejor opcion segun la \textbf{\textit{Ecuacion de Recurrencia}} de nuestro problema.

Para definir nuestra ecuacion de recurrencia, planteamos lo siguiente:
\begin{itemize}
    \item Para el primer dia, la ganancia va a ser el minimo entre la energia de ese dia y el valor del entrenamiento.
    \item Para el segundo dia va a depender de si conviene haber entrenado el primer dia y obtener la ganancia del entrenamiento como si fuera el primer dia de haber entrenado o continuar entrenando despues del primer dia pero \textbf{no} con la energia que tendrian disponible los jugadores si hubieran descansado.
    \item Para el entrenamiento $e_i$, el optimo seria el que maximiza la ganancia entre haber entrenado el dia $e_{i-1}$ y tener menos energia disponible para obtener la ganancia de ese dia, o no haber entrenado el dia anterior y tener la posibilidad de obtener mas ganancia teniendo mas energia disponible.
\end{itemize}
Entonces teniendo todo esto en mente, nuestra ecuacion de recurrencia tendria la siguiente forma:

INSERTAR FOTO ECUACION RECURRENCIA!!!!!!!!!!!!!!!!!!!

