\section{Algoritmo para encontrar la mayor ganancia en los entrenamientos}

A continuación se detallan el código y los pasos que se siguieron para llevar a cabo el algoritmo planteado utilizando \textit{Programacion Dinamica}.

\subsection{Obtener el cronograma}

Una vez que se obtuvieron los datos para los entrenamientos y la energia disponible dependiendo la cantidad de dias seguidos de haber entrenado, se arma una matriz de tamanio $NxN$ siendo $N$ la cantidad de dias en el cronograma:

\lstinputlisting[language=Python]{code/workout.py}


## POR ACA



La complejidad del algoritmo para ordenar todos los datos es $\mathcal{O}\left(n \log n\right)$, ya que el método \textit{sort} de Python tiene esa complejidad (\href{https://www.geeksforgeeks.org/sort-in-python/}{complejidad del método \textit{sort}}) y la función auxiliar que usa como \textit{key} para ordenar es $\mathcal{O}(1)$. Luego, recorrer todo los datos ya ordenados e ir procesando la información es $\mathcal{O}(n)$, ya que solo recorre el arreglo y va sumando los valores correspondientes a las variables declaradas, y eso es $\mathcal{O}(1)$. La complejidad final es $\mathcal{O}\left(n \log n + n\right)$.

La variable \textit{total} se utiliza para ir contabilizando el tiempo que tarda Scaloni, sumando el tiempo del video que está analizando en el momento más todos los ya analizados. A su vez, tenemos la variable \textit{actual}, que se encarga de guardar el tiempo que va a requerir el video que se está analizando actualmente, que es la suma del tiempo que se lleva en total más la que vaya a tardar el ayudante. Y por último tenemos la variable \textit{longest}, que se encarga de almacenar el video que vaya a tardar más en analizarse. Es necesaria ya que nada nos garantiza que el último ayudante vaya a ser el que más influencia en el tiempo total de los análisis de los videos. Un ejemplo para mostrar esto es ir a la exageración: Que el primer video de todos, el ayudante tarde 5 meses. Por más que el último tarde 2 hs, la duración total va a ser lo que tarde el analisis que más se aleje de los ya terminados, tanto por Scaloni y los demás ayudantes.
