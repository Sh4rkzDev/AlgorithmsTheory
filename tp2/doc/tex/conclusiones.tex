\section{Conclusiones}

El algoritmo planteado y la forma en la que se diseñó la solución pudo ser llevada a cabo de forma efectiva y encontrar la forma óptima de resolver el problema. Ahora bien, el hecho de que en vez de haber usado un simple arreglo como estructura de datos para resolver el problema nunca hubiera sido acorde a la solución. Como se explicó, no solo hay que tener en cuenta la ganancia del entrenamiento $e_i$, sino también la energía que corresponda a ese entrenamiento, y encontrar de esta manera la óptima planificación para obtener la mayor ganancia posible.

Por otro lado, se pudo también cumplir con las ideas planteadas sobre la \textit{Programación Dinámica}, el uso de la memorización para recordar subproblemas ya resueltos y aprovechar estos para resolver problemas mayores hasta llegar a la solución óptima. De igual manera, se pudo crear el algoritmo para reconstruir el cronograma a partir de la matriz construida y poder si se lo desea obtener los días a entrenar y los días a descansar.


Finalmente, una vez más, valoramos la oportunidad que se nos brindó de colaborar en esta tarea crucial para que los jugadores tengan el mejor desempeño y estamos determinados a contribuir el éxito continuo de la selección \textbf{CAMPEONA DEL MUNDO}.

